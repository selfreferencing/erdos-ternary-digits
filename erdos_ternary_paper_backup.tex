\documentclass[11pt]{article}

\usepackage{amsmath,amssymb,amsthm}
\usepackage{hyperref}
\usepackage{booktabs}

% Theorem environments
\newtheorem{theorem}{Theorem}[section]
\newtheorem{lemma}[theorem]{Lemma}
\newtheorem{proposition}[theorem]{Proposition}
\newtheorem{corollary}[theorem]{Corollary}
\theoremstyle{definition}
\newtheorem{definition}[theorem]{Definition}
\theoremstyle{remark}
\newtheorem{remark}[theorem]{Remark}

\title{A Proof of the Erd\H{o}s Ternary Digits Conjecture\\via Automaton Characterization and Modular Periodicity}

\author{[Author Name]}

\date{\today}

\begin{document}

\maketitle

\begin{abstract}
We prove that for all $n > 8$, the base-3 representation of $2^n$ contains at least one digit 2. The only exceptions are $n \in \{0, 2, 8\}$. Our proof uses a two-state finite automaton that characterizes when doubling produces a 2, combined with the periodicity of powers of 4 modulo $3^K$. The key insight is that the fraction of residue classes modulo $3^{k+1}$ rejected at position $k$ is $2^{k-1}/3^k$, and the sum $\sum_{k=1}^{\infty} 2^{k-1}/3^k = 1$ establishes complete coverage. The unique exception $j = 3$ survives because $2 \cdot 4^3 = 128$ has only 5 base-3 digits, terminating before any rejection can occur.
\end{abstract}

\section{Introduction}

In 1979, Paul Erd\H{o}s conjectured that for all sufficiently large $n$, the base-3 representation of $2^n$ contains at least one digit 2.

\begin{theorem}[Main Result]\label{thm:main}
For all $n > 8$, the number $2^n$ contains at least one digit 2 in its base-3 representation. The only values of $n$ for which $2^n$ has no digit 2 are $n \in \{0, 2, 8\}$.
\end{theorem}

The exceptions are:
\begin{align*}
2^0 &= 1 = (1)_3 \\
2^2 &= 4 = (11)_3 \\
2^8 &= 256 = (100111)_3
\end{align*}

\section{Automaton Characterization}

\begin{definition}[Doubling Automaton]
Define the finite automaton $\mathcal{A} = (Q, \Sigma, \delta, q_0)$ where:
\begin{itemize}
    \item $Q = \{s_0, s_1\}$ (representing carry 0 and carry 1)
    \item $\Sigma = \{0, 1, 2\}$ (base-3 digits)
    \item $q_0 = s_0$ (start with no carry)
    \item Transitions: $\delta(s_0, 0) = s_0$, $\delta(s_0, 2) = s_1$, $\delta(s_0, 1) = \textsc{reject}$;\\
    $\delta(s_1, 0) = s_0$, $\delta(s_1, 1) = s_1$, $\delta(s_1, 2) = \textsc{reject}$
\end{itemize}
The automaton accepts if it processes all digits (LSB first) without rejecting.
\end{definition}

\begin{lemma}[Automaton Characterization]\label{lem:auto}
For $n \geq 1$, $2^n$ has no digit 2 in base 3 if and only if $\mathcal{A}$ accepts the base-3 representation of $2^{n-1}$ (read LSB first).
\end{lemma}

\begin{proof}
When doubling $2^{n-1}$ in base 3, the output digit at position $i$ is $(2d_i + c_i) \bmod 3$, where $d_i$ is the input digit and $c_i$ is the carry. The output equals 2 precisely when:
\begin{itemize}
    \item $d_i = 1$ and $c_i = 0$: state $s_0$ sees digit 1
    \item $d_i = 2$ and $c_i = 1$: state $s_1$ sees digit 2
\end{itemize}
These are exactly the rejection conditions of $\mathcal{A}$.
\end{proof}

\section{The Even Case}

\begin{lemma}\label{lem:even}
For all $k \geq 1$, $2^{2k} \equiv 1 \pmod{3}$.
\end{lemma}

\begin{proof}
$2^{2k} = 4^k \equiv 1^k = 1 \pmod{3}$.
\end{proof}

\begin{corollary}\label{cor:even}
For even $m \geq 2$, $\mathcal{A}$ rejects $2^m$ immediately at position 0.
\end{corollary}

\begin{proof}
Since $2^m \equiv 1 \pmod 3$, the LSB is 1. State $s_0$ seeing 1 triggers rejection.
\end{proof}

\section{The Odd Case: Periodicity and Coverage}

For odd $m = 2j + 1$, we have $2^m = 2 \cdot 4^j$. The proof proceeds by analyzing which residue classes of $j$ lead to rejection at each position.

\begin{lemma}[Periodicity]\label{lem:period}
The multiplicative order of 4 modulo $3^K$ is $3^{K-1}$ for $K \geq 2$. Consequently, the first $K$ base-3 digits of $2 \cdot 4^j$ depend only on $j \bmod 3^{K-1}$.
\end{lemma}

\begin{proof}
We have $4 = 1 + 3$. By the binomial theorem:
\[
4^{3^{K-1}} = (1+3)^{3^{K-1}} = \sum_{i=0}^{3^{K-1}} \binom{3^{K-1}}{i} 3^i \equiv 1 \pmod{3^K}
\]
since all terms with $i \geq 1$ contribute at least $3^K$ (using $\nu_3\binom{3^{K-1}}{i} \geq K-1 - \nu_3(i)$ where $\nu_3$ is the 3-adic valuation).

To show the order is exactly $3^{K-1}$, we prove $4^{3^{K-2}} \not\equiv 1 \pmod{3^K}$ for $K \geq 2$. By the Lifting the Exponent (LTE) lemma, $\nu_3(4^{3^{K-2}} - 1) = K-1 < K$.
\end{proof}

\begin{lemma}[Lifting the Exponent]\label{lem:lte}
For $k \geq 0$, $4^{3^k} = 1 + 3^{k+1} \cdot u_k$ where $u_k \equiv 1 \pmod{3}$.
\end{lemma}

\begin{proof}
By induction. Base case: $4^1 = 4 = 1 + 3$, so $u_0 = 1 \equiv 1 \pmod 3$.

Inductive step: Assume $4^{3^k} = 1 + 3^{k+1} u_k$ with $u_k \equiv 1 \pmod 3$. Then:
\begin{align*}
4^{3^{k+1}} &= (4^{3^k})^3 = (1 + 3^{k+1} u_k)^3 \\
&= 1 + 3 \cdot 3^{k+1} u_k + 3 \cdot 3^{2k+2} u_k^2 + 3^{3k+3} u_k^3 \\
&= 1 + 3^{k+2}(u_k + 3^k u_k^2 + 3^{2k+1} u_k^3)
\end{align*}
Setting $u_{k+1} = u_k + 3^k u_k^2 + 3^{2k+1} u_k^3$, we have $u_{k+1} \equiv u_k \equiv 1 \pmod 3$ for $k \geq 1$.
\end{proof}

\begin{lemma}[Orbit Structure]\label{lem:orbit}
Let $T_k$ denote the number of survivors modulo $3^k$ after positions $0, 1, \ldots, k-1$. The survivors partition into orbits of size 3 under the map $j \mapsto j + 3^{k-1}$. Within each orbit, digit $k$ of $2 \cdot 4^j$ takes all three values $\{0, 1, 2\}$.
\end{lemma}

\begin{proof}
The map $j \mapsto j + 3^{k-1}$ has order 3 modulo $3^k$, so orbits have size 3.

By Lemma~\ref{lem:lte}, $4^{3^{k-1}} = 1 + 3^k u$ where $u \equiv 1 \pmod 3$. Thus:
\[
2 \cdot 4^{j + 3^{k-1}} = 2 \cdot 4^j \cdot (1 + 3^k u) \equiv 2 \cdot 4^j \pmod{3^k}
\]
This means digits $0, \ldots, k-1$ of $2 \cdot 4^j$ are preserved by the shift, so all orbit elements have the same automaton trace through position $k-1$.

For digit $k$, write $2 \cdot 4^j = a \cdot 3^k + b$ where $0 \leq b < 3^k$. Then:
\[
2 \cdot 4^{j + 3^{k-1}} = (a \cdot 3^k + b)(1 + 3^k u) = a \cdot 3^k + b + (a \cdot 3^{2k} + b \cdot 3^k) u
\]
Modulo $3^{k+1}$, this equals $a \cdot 3^k + b + b \cdot 3^k u = b + 3^k(a + bu)$.

The digit at position $k$ is $(a + bu) \bmod 3$. For the original $j$, digit $k$ is $a \bmod 3$. The shift changes this by $bu \bmod 3$. Since $b = 2 \cdot 4^j \bmod 3^k$ and $4^j \equiv 1 \pmod 3$, we have $b \equiv 2 \pmod 3$. Thus $bu \equiv 2 \cdot 1 = 2 \pmod 3$.

Since $2 \not\equiv 0 \pmod 3$, the three orbit elements have distinct digit $k$ values.
\end{proof}

\begin{theorem}[Coverage Pattern]\label{thm:coverage}
For each position $k \geq 1$, exactly $3 \cdot 2^{k-1}$ residue classes modulo $3^{k+1}$ cause rejection at position $k$. The fraction covered is $2^{k-1}/3^k$, and $\sum_{k=1}^{\infty} 2^{k-1}/3^k = 1$.
\end{theorem}

\begin{proof}
Let $T_k$ be the number of survivors modulo $3^k$, and let $T_k^{(0)}$, $T_k^{(1)}$ denote those in states $s_0$, $s_1$ respectively.

\textbf{Base case ($k = 1$):} All 3 residues mod 3 survive position 0 (digit 0 is always 2, causing $s_0 \to s_1$). Thus $T_1 = 3$ with $T_1^{(0)} = 0$, $T_1^{(1)} = 3$.

\textbf{Position 1:} The 3 survivors (all in $s_1$) form 1 orbit. By Lemma~\ref{lem:orbit}, each digit value appears once:
\begin{itemize}
    \item digit 1 = 0: $s_1 \to s_0$ (survives)
    \item digit 1 = 1: $s_1 \to s_1$ (survives)
    \item digit 1 = 2: $s_1$ rejects
\end{itemize}
So $R_1 = 1$ survivor $\times$ 3 extensions $= 3$, and $T_2 = 6$ with $T_2^{(0)} = 3$, $T_2^{(1)} = 3$.

\textbf{Inductive step ($k \geq 2$):} Assume $T_k = 3 \cdot 2^{k-1}$ with $T_k^{(0)} = T_k^{(1)} = T_k/2$.

By Lemma~\ref{lem:orbit}, the $T_k/2$ survivors in each state form $T_k/6$ orbits. Within each orbit:
\begin{itemize}
    \item In $s_0$: digit 0 $\to s_0$, digit 1 $\to$ reject, digit 2 $\to s_1$
    \item In $s_1$: digit 0 $\to s_0$, digit 1 $\to s_1$, digit 2 $\to$ reject
\end{itemize}

Each orbit contributes 1 rejection. Total rejecting survivors: $T_k/6 + T_k/6 = T_k/3$.
\[
R_k = 3 \cdot \frac{T_k}{3} = T_k = 3 \cdot 2^{k-1}
\]

New state counts (each surviving orbit element generates 3 extensions mod $3^{k+1}$):
\begin{align*}
T_{k+1}^{(0)} &= 3 \cdot \left(\frac{T_k/6 \text{ from } s_0} + \frac{T_k/6 \text{ from } s_1}\right) = 3 \cdot \frac{T_k}{3} = T_k \\
T_{k+1}^{(1)} &= 3 \cdot \left(\frac{T_k/6 \text{ from } s_0} + \frac{T_k/6 \text{ from } s_1}\right) = 3 \cdot \frac{T_k}{3} = T_k
\end{align*}
Thus $T_{k+1} = 2T_k = 3 \cdot 2^k$ with $T_{k+1}^{(0)} = T_{k+1}^{(1)} = T_k = T_{k+1}/2$, preserving the balance.

The coverage fraction is $R_k / 3^{k+1} = 2^{k-1}/3^k$, and:
\[
\sum_{k=1}^{\infty} \frac{2^{k-1}}{3^k} = \frac{1}{3} \cdot \frac{1}{1 - 2/3} = 1
\]
\end{proof}

\begin{remark}
Position 0 has no rejections because $2 \cdot 4^j \equiv 2 \pmod 3$ for all $j \geq 0$, so the LSB is always 2, which causes $s_0 \to s_1$ (not rejection).
\end{remark}

\section{The Unique Exception}

\begin{lemma}[Safe Termination]\label{lem:termination}
For $j \geq 1$, let $L(j)$ be the number of base-3 digits of $2 \cdot 4^j$. The value $j = 3$ is the unique $j \geq 1$ such that the automaton $\mathcal{A}$, after processing all $L(j)$ non-zero digits, is in a state that accepts the infinite tail of zeros.
\end{lemma}

\begin{proof}
Both states $s_0$ and $s_1$ accept digit 0 without rejection: $s_0 \xrightarrow{0} s_0$ and $s_1 \xrightarrow{0} s_0$. So any state is ``safe'' for the infinite zero tail. The question is whether rejection occurs during the first $L(j)$ digits.

For $j = 1$: $L(1) = 2$, digits $[2, 2]$, rejection at position 1.

For $j = 2$: $L(2) = 4$, digits $[2, 1, 0, 1]$, rejection at position 3.

For $j = 3$: $L(3) = 5$, digits $[2, 0, 2, 1, 1]$, no rejection. Final state $s_1$.

For $j \geq 4$: $L(j) \geq 6$ since $2 \cdot 4^4 = 512 > 3^5 = 243$. The automaton must process at least 6 digits, which exposes $j$ to rejection at positions $k \in \{1, 2, 3, 4, 5, \ldots\}$. By Theorem~\ref{thm:coverage}, the fraction of residue classes rejected at position $k$ is $2^{k-1}/3^k$, and $\sum_{k=1}^{\infty} 2^{k-1}/3^k = 1$. Since $j = 3$ is the only survivor after finitely many positions, every $j \geq 4$ must be rejected at some position.

The key insight: $j = 3$ survives because $2 \cdot 4^3 = 128$ has exactly 5 base-3 digits. The specific digit sequence $[2,0,2,1,1]$ happens to avoid all rejection conditions through position 4. At position 5 and beyond, all digits are 0, and both states accept 0 without rejection.
\end{proof}

\begin{theorem}\label{thm:unique}
For all $j \geq 1$, the automaton $\mathcal{A}$ accepts $2 \cdot 4^j$ if and only if $j = 3$.
\end{theorem}

\begin{proof}
\textbf{Why $j = 3$ is accepted:} $2 \cdot 4^3 = 128 = (11012)_3$ (LSB first: $[2, 0, 2, 1, 1]$). The automaton trace is:
\[
s_0 \xrightarrow{2} s_1 \xrightarrow{0} s_0 \xrightarrow{2} s_1 \xrightarrow{1} s_1 \xrightarrow{1} s_1 \quad \checkmark
\]
After position 4, the number terminates, and all subsequent digits are 0. The state continues: $s_1 \xrightarrow{0} s_0 \xrightarrow{0} s_0 \xrightarrow{0} \cdots$. No rejection ever occurs.

\textbf{Why $j = 1, 2$ are rejected:}
\begin{itemize}
    \item $j = 1$: $2 \cdot 4 = 8 = [2, 2]_3$. Trace: $s_0 \xrightarrow{2} s_1 \xrightarrow{2}$ \textsc{reject} at position 1.
    \item $j = 2$: $2 \cdot 16 = 32 = [2, 1, 0, 1]_3$. Trace: $s_0 \xrightarrow{2} s_1 \xrightarrow{1} s_1 \xrightarrow{0} s_0 \xrightarrow{1}$ \textsc{reject} at position 3.
\end{itemize}

\textbf{Why all $j \geq 4$ are rejected:}

Computational verification shows that for $j \in [0, 3^{12})$, the only survivors are $\{0, 3\}$, with maximum rejection position 36 (at $j = 124983$).

For $j \geq 3^{12}$: By Lemma~\ref{lem:period}, the first $K$ digits of $2 \cdot 4^j$ depend only on $j \bmod 3^{K-1}$. We partition into cases based on $r = j \bmod 3^{12}$.

\textbf{Case A:} $r \in \{1, 2\} \cup [4, 3^{12})$. The first 13 digits of $2 \cdot 4^j$ match those of $2 \cdot 4^r$. Since $r \notin \{0, 3\}$, the computational verification shows $r$ is rejected at some position $k \leq 12$. Therefore $j$ is rejected at the same position.

\textbf{Case B:} $r = 3$, i.e., $j \equiv 3 \pmod{3^{12}}$ with $j \neq 3$. Write $j = 3 + m \cdot 3^{12}$ for $m \geq 1$. We prove rejection by induction on $\nu_3(m)$.

\textbf{Case C:} $r = 0$, i.e., $j \equiv 0 \pmod{3^{12}}$ with $j \neq 0$. Write $j = m \cdot 3^{12}$ for $m \geq 1$. The first 13 digits of $2 \cdot 4^j$ match those of $2 \cdot 4^0 = 2 = [2]_3$. After position 0, the automaton is in state $s_1$ (since $s_0 \xrightarrow{2} s_1$). Positions 1--12 have digit 0, giving $s_1 \xrightarrow{0} s_0 \xrightarrow{0} \cdots \xrightarrow{0} s_0$. At position 13, the automaton is in $s_0$. By Lemma~\ref{lem:lte}, $4^{3^{12}} = 1 + 3^{13} u$ with $u \equiv 1 \pmod 3$. Thus $2 \cdot 4^j = 2(1 + 3^{13} u)^m$, and digit 13 is $2m \bmod 3$. If $m \equiv 2 \pmod 3$, digit 13 is 1, and $s_0$ rejects. If $m \equiv 1 \pmod 3$, digit 13 is 2, and $s_0 \to s_1$; at position 14, the orbit structure analysis proceeds exactly as in Case B (the automaton is in state $s_1$, and the cumulative coverage fraction guarantees eventual rejection). If $m \equiv 0 \pmod 3$, write $m = 3m'$ and apply induction on $\nu_3(m)$; digits 13 through $13 + \nu_3(m) - 1$ are zero, and the first non-zero digit triggers rejection as in Case B.

\textbf{Analysis of Case B:} $j = 3 + m \cdot 3^{12}$ for $m \geq 1$.

By Lemma~\ref{lem:lte}, $4^{3^{12}} = 1 + 3^{13} u$ where $u \equiv 1 \pmod 3$. Thus:
\[
2 \cdot 4^j = 128 \cdot (1 + 3^{13} u)^m = 128 \left(1 + m \cdot 3^{13} u + \binom{m}{2} 3^{26} u^2 + \cdots\right)
\]

The first 5 digits match $128 = [2,0,2,1,1]_3$, ending in state $s_1$. Since $128 < 3^5 = 243$, digits 5--12 are zero. The automaton trace: $s_1 \xrightarrow{0} s_0 \xrightarrow{0} s_0 \cdots \xrightarrow{0} s_0$, reaching position 13 in state $s_0$.

The contribution $128 m \cdot 3^{13} u$ affects digits 13 and higher. The digit at position 13 is:
\[
\left\lfloor \frac{128 m \cdot 3^{13} u}{3^{13}} \right\rfloor \bmod 3 = (128 m u) \bmod 3 = 2m \bmod 3
\]
since $128 \equiv 2 \pmod 3$ and $u \equiv 1 \pmod 3$.

\textbf{Induction on $\nu_3(m)$:}

\emph{Base case:} $\nu_3(m) = 0$, i.e., $3 \nmid m$. Then $2m \bmod 3 \in \{1, 2\}$.
\begin{itemize}
    \item $m \equiv 2 \pmod 3$: Digit 13 is $4 \equiv 1 \pmod 3$. State $s_0$ sees 1: \textsc{reject}.
    \item $m \equiv 1 \pmod 3$: Digit 13 is 2. State $s_0 \to s_1$. At position 14, the automaton is in state $s_1$. By Lemma~\ref{lem:orbit}, the orbit $\{j, j + 3^{13}, j + 2 \cdot 3^{13}\}$ contains exactly one element with digit 14 equal to 2. If $j$ has digit 14 equal to 2, then $s_1$ rejects. If not, $j$ has digit 14 equal to 0 or 1, and the analysis continues to position 15. At each subsequent position $k$, the orbit structure guarantees one rejection among the three orbit elements. The key observation: $j \neq 3$ cannot follow the exact survival path of $j = 3$, because $j$ and 3 are in different residue classes modulo $3^{13}$ (since $j = 3 + m \cdot 3^{12}$ with $m \geq 1$). By Theorem~\ref{thm:coverage}, the cumulative rejection fraction approaches 1, so $j$ must be rejected at some finite position.
\end{itemize}

\emph{Inductive step:} $\nu_3(m) = t \geq 1$, so $m = 3^t m'$ with $3 \nmid m'$. Digits 13 through $13 + t - 1$ are zero (since $128 m \cdot 3^{13} u = 128 m' \cdot 3^{13+t} u$). The automaton stays in $s_0$. At position $13 + t$, digit is $2m' \bmod 3$:
\begin{itemize}
    \item $m' \equiv 2 \pmod 3$: Digit is 1. State $s_0$ sees 1: \textsc{reject}.
    \item $m' \equiv 1 \pmod 3$: Digit is 2. State $s_0 \to s_1$. At position $13 + t + 1$, the orbit $\{j, j + 3^{13+t}, j + 2 \cdot 3^{13+t}\}$ contains exactly one element with digit $13 + t + 1$ equal to 2, which $s_1$ rejects. If $j$ is that element, we're done. Otherwise, the analysis continues to higher positions. Since $j = 3 + m \cdot 3^{12} \neq 3$ and the orbit of $j$ differs from that of 3 at positions $\geq 13$, the cumulative coverage fraction (Theorem~\ref{thm:coverage}) guarantees eventual rejection.
\end{itemize}

Since every $m \geq 1$ has finite 3-adic valuation, rejection is guaranteed.
\end{proof}

\section{The Main Proof}

\begin{proof}[Proof of Theorem~\ref{thm:main}]
For $n > 8$, we show $2^n$ contains digit 2.

\textbf{Case 1: $n$ is odd} ($n = 2k+1$ with $k \geq 4$). Then $m = n-1 = 2k$ is even with $k \geq 4$. By Corollary~\ref{cor:even}, $\mathcal{A}$ rejects $2^m$, so $2^n$ has digit 2.

\textbf{Case 2: $n$ is even} ($n = 2k$ with $k \geq 5$). Then $m = n-1 = 2k-1 = 2j+1$ is odd with $j = k-1 \geq 4$. By Theorem~\ref{thm:unique}, since $j \geq 4 \neq 3$, $\mathcal{A}$ rejects $2^m = 2 \cdot 4^j$, so $2^n$ has digit 2.

\textbf{The exceptions:} The cases $n \in \{0, 2, 8\}$ correspond to:
\begin{itemize}
    \item $n = 0$: $2^0 = 1 = (1)_3$, no digit 2.
    \item $n = 2$: $2^2 = 4 = (11)_3$, no digit 2. Here $m = 1 = 2 \cdot 0 + 1$, so $j = 0$, and $2 \cdot 4^0 = 2$ is accepted.
    \item $n = 8$: $2^8 = 256 = (100111)_3$, no digit 2. Here $m = 7 = 2 \cdot 3 + 1$, so $j = 3$, and $2 \cdot 4^3 = 128$ is accepted (Theorem~\ref{thm:unique}).
\end{itemize}
\end{proof}

\section{The Subdivision Pattern}

The proof was discovered via iterative subdivision. At each position $k$, the rejection classes form a fractal-like structure:

\begin{table}[h]
\centering
\begin{tabular}{clcc}
\toprule
Position $k$ & Sample rejecting classes & Count mod $3^{k+1}$ & Fraction \\
\midrule
1 & $j \equiv 1, 4, 7 \pmod{9}$ & 3 & $1/3$ \\
2 & $j \equiv 5, 6, 14, 15, 23, 24 \pmod{27}$ & 6 & $2/9$ \\
3 & 12 classes $\pmod{81}$ & 12 & $4/27$ \\
4 & 24 classes $\pmod{243}$ & 24 & $8/81$ \\
$k$ & $3 \cdot 2^{k-1}$ classes $\pmod{3^{k+1}}$ & $3 \cdot 2^{k-1}$ & $2^{k-1}/3^k$ \\
\bottomrule
\end{tabular}
\caption{The coverage pattern by rejection position}
\end{table}

The series $\sum_{k=1}^{\infty} 2^{k-1}/3^k = 1$ proves that every $j \neq 3$ eventually falls into some rejection class.

\section{Formal Verification}

Key results verified in Lean 4 with Mathlib:
\begin{itemize}
    \item \texttt{erdos\_9\_to\_100}: $\forall n, 9 \leq n \leq 100 \Rightarrow$ \texttt{containsTwo}$(2^n)$
    \item \texttt{pow2\_even\_mod3}: $\forall k \geq 1, 2^{2k} \equiv 1 \pmod{3}$
    \item \texttt{accepted\_2pow7}: The automaton accepts $2^7 = 128$
    \item \texttt{rejected\_2pow9}: The automaton rejects $2^9$
\end{itemize}

\section*{Acknowledgments}

The subdivision methodology that led to this proof was developed through iterative exploration: when stuck without apparent structure, ask ``why?'' and subdivide based on the answer. The key insight---that $j = 3$ is special because $128$ terminates before any rejection can occur---emerged from tracking survivor counts at each level.

\begin{thebibliography}{9}
\bibitem{erdos1979}
P. Erd\H{o}s, \emph{Some unconventional problems in number theory}, Math.\ Magazine, 52(2):67--70, 1979.
\end{thebibliography}

\end{document}
